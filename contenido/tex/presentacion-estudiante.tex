
\chapter*{Información personal y académica}
\addcontentsline{toc}{chapter}{Información personal y académica}
\markboth{Información personal y académica}{Información personal y académica}


%%%%%%%%%%%%%%%%%%%%%%%%%%%%%%%%%%%%%%%%%%%%%%%%%%%%%%%%%%%%%%%%%%%%%%
% Llene todos los campos, respetando tildes, mayúsculas y minúsculas.
\section*{Datos personales}

\begin{description}
\item[{Nombre completo}] Javier Esteban Chandía Valdés % nombres y apellidos completos.
\item[{Matrícula}] 2023422819               % matrícula udec
\item[{Fecha de Nacimiento}] 24 de Junio, 2004 % día de mes de año
\item[{Nacionalidad}] Chileno
\item[{E-Mail institucional}] \href{mailto:jachandia2023@udec.cl}{jachandia2023@udec.cl}
\end{description}


%%%%%%%%%%%%%%%%%%%%%%%%%%%%%%%%%%%%%%%%%%%%%%%%%%%%%%%%%%%%%%%%%%%%%%
\section*{Breve biografía académica}
% Redacte una breve biografía (5 a 7 líneas) que incluya los
% siguientes aspectos:
% - Su nombre completo y el año en el que ingresó a la Universidad de
% Concepción.
% - Mencione su carrera actual y en qué año académico se encuentra.
% - Describa brevemente su trayectoria educativa previa a la universidad
% (por ejemplo, dónde cursó la educación media y cualquier logro académico
% relevante).
% - Mencione sus metas académicas y profesionales al finalizar el
% pregrado. ¿Qué le gustaría lograr al terminar la carrera? ¿En qué
% áreas le gustaría especializarse o trabajar?
% - Si lo considera pertinente, puede mencionar cualquier actividad
% extracurricular que haya contribuido a su formación (cursos,
% proyectos, trabajos, etc.).

Soy Javier Chandía, estudiante de segundo año de la carrera Ciencias Físicas.
Ingresé el año 2023 y actualmente estoy cursando el 4to semestre de la carrera Ciencias Físicas.

 La educación media la realicé en el colegio San Cristobal de la comuna de Talcahuano.
 Participé de las olimpiadas de física (2022) a nivel interregional (Bio-bio) y nacional (Organizado por \href{https://sochifi2022.com/}{sochifi2022}).

Actualmente, tengo la meta de graduarme con honores de pregrado, con el objetivo de seguir en la academia, investigando diferentes áreas de interes afín a la física (sin acotarse a ninguna en particular).



%%%%%%%%%%%%%%%%%%%%%%%%%%%%%%%%%%%%%%%%%%%%%%%%%%%%%%%%%%%%%%%%%%%%%%
\section*{Visión general e interés sobre la asignatura}
% En esta sección, reflexione y describa:
\subsubsection{Preguntas y Respuestas}
\begin{enumerate}[start=1, label={\bfseries \arabic*})]
\item ¿Cuál es su percepción inicial sobre la asignatura de Física Computacional II?
\end{enumerate}

Inicialmente, no estaba seguro de como iba a ser el formato de presentación del contenido del curso. Una vez transcurrido un intervalo de tiempo, realicé que Física Computacional II es una asignatura de dedicación constante--pues, si bien los contenidos iniciales son sencillos, la profundidad y el esfuerzo del trabajo incrementa a medida que uno le dedica tiempo.
\begin{enumerate}[start=2, label={\bfseries \arabic*})]
\item ¿Cómo se relaciona con su formación académica y sus intereses?
\end{enumerate}

Cuando inicié la carrera, tenía la percepción (alterada) de que los físicos trabajan exclusivamente con un lápiz y una pizarra.\\
Actualmente, sé que la complejidad de ciertos modelos físicos son solo solubles computacionalmente, más no analíticamente. Física computacional II es necesaria, a nivel de formación, para producir como físico, en todas las áreas de la física.
\begin{enumerate}[start=3, label={\bfseries \arabic*})]
\item ¿Qué habilidades o conocimientos espera desarrollar en esta asignatura, específicamente en el uso de herramientas computacionales aplicadas a la física?
\end{enumerate}

Espero desarrollar métodos de cálculo numérico, estimar errores computacionales, desarrollar habilidades blandas en la creación de artículos científicos y/o reportes.
\begin{enumerate}[start=4, label={\bfseries \arabic*})]
\item ¿De qué manera cree que lo aprendido en esta asignatura contribuirá a su desempeño en otros cursos o en su carrera profesional a futuro?
\end{enumerate}

Opino que esta asignatura es tan necesaria para un físico como para un ajedrecista aprender estructura de peones; puedes pasar toda la vida sin entenderla, y usandola a nivel de superficie, más los fundamentos desarrollados son los que permiten avanzar a un nivel más alto de forma profesional
% - % - Si tiene alguna expectativa específica o tema de interés particular dentro de la asignatura, menciónelo aquí.


%%%%%%%%%%%%%%%%%%%%%%%%%%%%%%%%%%%%%%%%%%%%%%%%%%%%%%%%%%%%%%%%%%%%%%
\section*{Resultados esperados de este portafolio}
% En esta sección, reflexione sobre los resultados que espera obtener al
% realizar este portafolio. Puede incluir lo siguiente:
En este portafolio, espero haber desarrollado lo siguiente al finalizar el documento:
\begin{enumerate}[start=1, label={\bfseries \arabic*})]
\item Habilidades blandas en el desarrollo de documentos o artículos científicos
\item Estructura en la creación de documentos
\item Capacidad de ser más eficiente en el trabajo compartido
\item Resumir contenidos aprendidos en la asignatura, y consolidar conocimientos mediante la reescritura en un artículo.
\end{enumerate}

Además, este documento puede ser usado como referencia para trabajos futuros, tales como la Tesis, o algún artículo científico publicable en alguna revista.

La autoevaluación y la inclusión de evidencias es parte de lo que permite el progreso académico; no tener evidencias del trabajo no permite comunicarlo de forma convincente. 
La autoevaluación permite la autocrítica, la cual aumenta la calidad del documento y es una buena capacidad para recapitular el avance académico.
% - ¿Qué habilidades y conocimientos espera haber consolidado al completar este portafolio?

% - ¿Cómo cree que el portafolio le ayudará a organizar, analizar y aplicar los conceptos aprendidos durante la asignatura?

% - ¿De qué manera considera que este portafolio puede servirle como referencia o herramienta para su futura formación académica o profesional?

% - Reflexione sobre cómo el proceso de autoevaluación y la inclusión de evidencias le permitirá comprender mejor su propio progreso.